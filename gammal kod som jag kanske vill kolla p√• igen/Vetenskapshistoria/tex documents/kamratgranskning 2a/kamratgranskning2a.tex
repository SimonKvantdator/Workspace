\documentclass[a4paper, 11pt]{article}
\usepackage[utf8]{inputenc}
\usepackage[swedish]{babel}

\title{Vetenskapshistoria TEK130\\ Kamratgranskning II}
\author{}
\date{\today}

\begin{document}
\maketitle

Du har, i din text, redogort för vad \emph{big science} innebär och givit exempel på ett antal big science-projekt. Du har även diskuterat på vilka sätt vetenskapen influerades av militären under det kalla kriget. Din text är även sammanhängande och visar att du förstått litteraturen. Jag tycker också texten är snyggt inlett med ett citat.

Du har använt referenser bra genom texten. Snyggt med sidhänvisningarna också! Om något hade du kunnat referera till fler sidor samtidigt.

Jag tycker det går bra att hänga med i din text och det är tydligt vad du menar överlag. Det fanns dock några meningar, speciellt i första stycket, som inte var helt grammatiskt korrekta. Detta gör att texten ibland blir lite svår att läsa. Mycket av det grammatiska handlar säkert bara om att läsa igenom texten en gång till innan den är klar. \emph{Manhattanprojektet} ska skrivas ihop.




\end{document}