\documentclass[english, a4paper, 11pt]{article}


%%%%%%%%%%% LANGUAGE %%%%%%%%%%%

% For correct hyphenation in swedish
\usepackage[T1]{fontenc}

% For interpreting non-ASCII characters
% \usepackage[utf8]{inputenc}

% International language support
% Fetches language from documentclass options. Most other packages do this as well
\usepackage{babel}


%%%%%%%%%%% FORMAL STUFF %%%%%%%%%%%

% Smaller margins
\usepackage[margin=2.5cm]{geometry}

% Clickable urls
\usepackage[hyphens]{url} % Does not want to be loaded after physics package

% Fancy chapter headers
\usepackage{titlesec}
\titleformat{\chapter}{\normalfont\huge}{\thechapter.}{20pt}{\huge\it}

% Dates & time
\usepackage{datetime} % Useful when referencing websites

% What to display in table of contents
\setcounter{tocdepth}{3}
\setcounter{secnumdepth}{3}

% Lists
\usepackage{enumerate} % Determines the style in which the counter is printed
\usepackage{enumitem} % Provides user control over the layout of the three basic list environments

% Citing & bibliography
\usepackage{csquotes} % For \enquote command for proper quotation marks, also biblatex recommends this
\usepackage[backend=biber, style=numeric, sorting=none]{biblatex}
\bibliography{bibliography} % A file named bibliography.bib containing the bibTeX entries should be placed beside the main.tex file


%%%%%%%%%%% GRAPHICS %%%%%%%%%%%

\usepackage{graphics,color,xcolor}

% Figures
\usepackage{epsfig} % Solves some problems in \includegraphics{<.eps-file>}
\usepackage{graphicx} % More options for \includegraphics
\usepackage{wrapfig} % Figure environment that lets text wrap around figure
\usepackage{float} % Figure placement
\usepackage{caption} % More options for \caption
\usepackage{subcaption} % Subfigures

% Tikz
\usepackage{tikz}
\usepackage{pgf,pgfplots} % Pgfplot
\pgfplotsset{compat=1.15}

% För alduslöv
\usepackage{pifont}


%%%%%%%%%%% PHYSICS %%%%%%%%%%%

% SI units
\usepackage{siunitx}

% Physics macros
\usepackage{physics} % Defines lots of nice commands like \derivative, \norm, \evaluated, etc. It is recommended to use these as much as possible for nice spacing and readable LaTeX code.
\usepackage{braket} % Defines \bra, \ket, \braket, and \set
\usepackage{cancel} % Basically a better version of the \not command
\usepackage{slashed} % For Feynman slash notation
\usepackage{simpler-wick} % Wick contractions (may require sty-file)
\usepackage[compat=1.1.0]{tikz-feynman} % Feynman diagrams (has to be compiled with LuaTeX)


%%%%%%%%%%% CODING %%%%%%%%%%%

% For nice code insertions
\usepackage{listings}
\definecolor{codegreen}{rgb}{0,0.6,0}
\definecolor{codegray}{rgb}{0.5,0.5,0.5}
\definecolor{codepurple}{rgb}{0.58,0,0.82}
\definecolor{backcolour}{rgb}{0.95,0.95,0.92}
\lstdefinestyle{mystyle}{
    backgroundcolor=\color{backcolour},   
    commentstyle=\color{codegreen},
    keywordstyle=\color{magenta},
    numberstyle=\tiny\color{codegray},
    stringstyle=\color{codepurple},
    basicstyle=\ttfamily\footnotesize,
    breakatwhitespace=false,         
    breaklines=true,                 
    captionpos=b,                    
    keepspaces=true,                 
    numbers=left,                    
    numbersep=5pt,                  
    showspaces=false,                
    showstringspaces=false,
    showtabs=false,                  
    tabsize=2
}
\lstset{style=mystyle}


%%%%%%%%%%% MATHEMATICS %%%%%%%%%%%

% AMS packages
\usepackage{amsmath,amsfonts,amsthm,amssymb}

% Sats/bevismiljöer
\iflanguage{swedish}{
    \newtheorem{theorem}{Sats}
    \newtheorem*{theorem*}{Sats}
    \newtheorem{lemma}{Lemma}
    \newtheorem*{lemma*}{Lemma}
    \renewenvironment{proof}{\textit{Bevis.}}{\hfill$\square$}
    \theoremstyle{definition}
    \newtheorem{definition}{Definition}
    \newtheorem*{definition*}{Definition}
}{}
\iflanguage{english}{
    \newtheorem{theorem}{Theorem}
    \newtheorem*{theorem*}{Theorem}
    \newtheorem{lemma}{Lemma}
    \newtheorem*{lemma*}{Lemma}
    \renewenvironment{proof}{\textit{Proof.}}{\hfill$\square$}
    \theoremstyle{definition}
    \newtheorem{definition}{Definition}
    \newtheorem*{definition*}{Definition}
}{}

% Vectors are upright boldface. This definition is better than the physics package's \vectorbold I think.
\renewcommand{\Vec}[1]{{\boldsymbol{\mathrm{#1}}}}
\renewcommand{\vec}[1]{{\boldsymbol{\mathrm{#1}}}}

% Redefine \exp
% Errors occur if this definition is made before some of the packages are loaded
\let\oldexp\exp
\newcommand{\Exp}[1]{\oldexp{#1}}
\renewcommand{\exp}[1]{\mathrm{e}^{#1}}

% Some of my own shorthands for correct spacing in math environments
\newcommand{\from}{\colon} % Proper spacing of colon in functions f: A → B


%%%%%%%%%%% MISCELLANEOUS %%%%%%%%%%%

% In-pdf comments through \todo command
\setlength{\marginparwidth}{2cm} % Silence warning about margin size
\usepackage{todonotes}

% Clickable links and refs
\usepackage[hidelinks]{hyperref} 

% Cleverref automatically detects if you are referencing a figure, table, or equation etc
% Cleverref has to be loaded last I think, after babel and hyperref etc
\usepackage[noabbrev, nameinlink]{cleveref}
\crefname{equation}{}{}
\iflanguage{swedish}{ % Tell cleverref to use Oxford comma
	\newcommand{\creflastconjunction}{, och\nobreakspace}
}{}
\iflanguage{english}{
	\newcommand{\creflastconjunction}{, and\nobreakspace}
}{}


% Tag only referenced equations (this is not an ideal package, as it requires etextools, which is buggy and abandoned by its author)
% \expandafter\def\csname ver@etex.sty\endcsname{3000/12/31}
% \let\globcount\newcount
% \usepackage{autonum}

% For writing \overset{text}&{=} in align environment
\usepackage{aligned-overset} 