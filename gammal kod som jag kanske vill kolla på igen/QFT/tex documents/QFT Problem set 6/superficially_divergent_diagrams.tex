\begin{figure}[H]
    \centering
    \begin{subfigure}{.33\linewidth}
        \centering
        \feynmandiagram[small, inline=(o.base), vertical=a to o]{
        	o [blob],
        	i1 -- [draw=none] o,
            i2 -- [draw=none] o,
            i3 -- [draw=none] o,
            i4 -- [draw=none] o,
        };
        \caption{$D = 4$}
        \label{subfig:superficially_divergent_diagrams:a}
    \end{subfigure}%
    \begin{subfigure}{.33\linewidth}
        \centering
        \feynmandiagram[small, inline=(o.base), horizontal=o to a]{
        	o [blob],
        	a -- [scalar] o,
        	i1 -- [draw=none] o,
            i2 -- [draw=none] o,
            i3 -- [draw=none] o,
        };
        \caption{$D = 3$}
        \label{subfig:superficially_divergent_diagrams:b}
    \end{subfigure}%
    \begin{subfigure}{.33\linewidth}
        \centering
        \feynmandiagram[small, inline=(o.base), horizontal=a to o]{
        	o [blob],
        	i1 -- [draw=none] o,
        	i2 -- [draw=none] o,
            a -- [scalar] o,
            b -- [scalar] o,
        };
        \caption{$D = 2$}
        \label{subfig:superficially_divergent_diagrams:c}
    \end{subfigure}
    \begin{subfigure}{.33\linewidth}
        \centering
        \feynmandiagram[small, inline=(o.base), horizontal=o to a]{
        	o [blob],
        	a -- [scalar] o,
        	b -- [scalar] o,
            c -- [scalar] o,
        };
        \caption{$D = 1$}
        \label{subfig:superficially_divergent_diagrams:d}
    \end{subfigure}%
    \begin{subfigure}{.33\linewidth}
        \centering
        \feynmandiagram[small, inline=(o.base), vertical=a to b]{
        	o [blob],
        	a -- [scalar] o,
        	b -- [scalar] o,
        	a -- [draw=none] b,
            c -- [scalar] o,
            d -- [scalar] o,
        	c -- [draw=none] d,
        };
        \caption{$D = 0$}
        \label{subfig:superficially_divergent_diagrams:e}
    \end{subfigure}%
    \begin{subfigure}{.33\linewidth}
        \centering
        \feynmandiagram[small, inline=(o.base), horizontal=a to o]{
        	o [blob],
        	i1 -- [draw=none] o,
        	i2 -- [draw=none] o,
            o -- [fermion] a,
            b -- [fermion] o,
        };
        \caption{$D = 1$}
        \label{subfig:superficially_divergent_diagrams:f}
    \end{subfigure}
    \begin{subfigure}{.33\linewidth}
        \centering
        \feynmandiagram[small, inline=(o.base), vertical=a to o]{
        	o [blob],
            a -- [scalar] o,
            o -- [fermion] b,
            c -- [fermion] o,
        };
        \caption{$D = 0$}
        \label{subfig:superficially_divergent_diagrams:g}
    \end{subfigure}
    \caption{
%    We can translate the diagrams in figure 10.2 in Peskin to the pseduoscalar Yukawa theory by changing the photon propagators to scalar propagators. We have also added the arrows that are not written out in figure 10.2 (the directions of these arrows follow from momentum conservation TODO: DO I KNOW THIS IS TRUE?).
	}
    \label{fig:superficially_divergent_diagrams}
\end{figure}
