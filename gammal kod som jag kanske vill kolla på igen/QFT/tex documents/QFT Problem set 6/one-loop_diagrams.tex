\begin{figure}[H]
    \centering
    \begin{subfigure}{.33\linewidth}
		\centering
		\feynmandiagram[small, inline=(o1.base), horizontal=a to b]{
			a -- [scalar] o1,
			o1 -- [fermion, half left] o2,
			o2 -- [fermion, half left] o1,
			b -- [scalar] o2,
		};
		\setcounter{subfigure}{2}
		\caption{}
		\label{subfig:one-loop_diagrams:c}
	\end{subfigure}%
    \begin{subfigure}{.33\linewidth}
        \centering
        \feynmandiagram[small, inline=(o1.base), horizontal=o1 to o2]{
			a -- [scalar] o1,
			o1 -- [fermion, quarter right] o2,
			b -- [scalar] o2,
			o2 -- [fermion, quarter right] o3,
			c -- [scalar] o3,
			o3 -- [fermion, quarter right] o4,
			d -- [scalar] o4,
			o4 -- [fermion, quarter right] o1,
        };
		\setcounter{subfigure}{4}
        \caption{}
        \label{subfig:one-loop_diagrams:e}
    \end{subfigure}%
    \begin{subfigure}{.33\linewidth}
        \centering
        \feynmandiagram[small, inline=(o1.base), horizontal=a to b]{
            a -- [fermion] o1,
			o1 -- [fermion, half left] o2,
			o2 -- [scalar, half left] o1,
			o2 -- [fermion] b,
        };
        \caption{}
        \label{subfig:one-loop_diagrams:f}
    \end{subfigure}
    \begin{subfigure}{.33\linewidth}
        \centering
        \feynmandiagram[small, inline=(o.base), horizontal=o2 to o3]{
			a -- [fermion] o1,
			o1 -- [fermion, quarter right] o2,
			b -- [scalar] o2,
			o2 -- [fermion, quarter right] o3,
			o3 -- [fermion] c,
			o3 -- [scalar, quarter right] o1,
        };
        \caption{}
        \label{subfig:one-loop_diagrams:g}
    \end{subfigure}
    \caption{}
    \label{fig:one-loop_diagrams}
\end{figure}