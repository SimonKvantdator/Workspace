\documentclass[a4paper, 11pt]{article}
\usepackage[utf8]{inputenc}
\usepackage[swedish]{babel}

\title{Vetenskapshistoria TEK130\\ Kamratgranskning II}
\author{}
\date{\today}

\begin{document}
\maketitle

Du har, i din text, redogort för vad \emph{big science} innebär och givit exempel på ett antal big science-projekt. Du har även diskuterat på vilka sätt vetenskapen influerades av militären under det kalla kriget. Din text är även välskriven och visar att du har god förståelse för litteraturen. Jag tycker du är bra på att förmedla hur exemplen du tar upp är relevanta. Du har även en bra röd tråd genom texten. Dock hade du kunnat skriva någonstans i början av texten att det är begreppet \emph{big science} som du redogör för. Jag tror det hade gett lite mer kontext när du sedan använder begreppet senare i texten. Jag hade nog också tyckt att din text varit enklare att läsa om den var styckeindelad. Med blankrad eller inskjuten rad.






\end{document}
