\documentclass[a4paper, 11pt]{article}
\usepackage[utf8]{inputenc}
\usepackage[swedish]{babel}

\title{Vetenskapshistoria TEK130\\ Kamratgranskning I}
\author{}
\date{\today}

\begin{document}
\maketitle


I din text har du redogjort för de sätt att tänka kring biologi som växte fram under 1800-talet. Du har även tagit upp den förändrade synen på sexualitet och kön samt diskuterat hur kvinnors möjligheter att delta i naturvetenskaplig forskning förändrats under 1800- och 1900-talet. I början av din text formulerar du frågeställningen, vilket jag tycker är tydligt. Du svarar sen på frågorna med bra källhänvisningar. Du glömde dock bifoga källförteckningen i slutet av texten.

Texten är överlag lätt att läsa. Finns några meningar som inte är helt sammanhängande. Det handlar säkert bara om att läsa igenom texten en gång till efter att den är klar. Du har också använt ordet ''så'' på tre olika ställen. På varje ställe hade det kunnat skippas.

Ett förbättringsförslag hade kanske kunnat vara att binda ihop texten mer, till exempel mellan stycke tre och stycke fyra kunde jag inte följa någon röd tråd. Det hade också kanske varit intressant att ställa det sista påståendet i texten, ''mycket byggde på fördomar och att underliggande åsikter drev vilka slutsatserna var'', i kontrast till den nya vetenskapens fokus på experiment och syn på sig själv som antidogmatisk.





\end{document}