%! TEX root = /home/simon/Documents/masterarbete/dagbok/main.tex
\documentclass[swedish, a4paper, 11pt]{report}


%%%%%%%%%%% LANGUAGE %%%%%%%%%%%

% For correct hyphenation in swedish
\usepackage[T1]{fontenc}

% For interpreting non-ASCII characters
\usepackage[utf8]{inputenc}

% International language support
% Fetches language from documentclass options. Most other packages do this as well
\usepackage{babel}


%%%%%%%%%%% FORMAL STUFF %%%%%%%%%%%

% Smaller margins
\usepackage[margin=2.5cm]{geometry}

% Fancy chapter headers
\usepackage{titlesec}
\titleformat{\chapter}{\normalfont\huge}{\thechapter.}{20pt}{\huge\it}

% Dates & time
\usepackage[yyyymmdd]{datetime} % Useful when referencing websites
\renewcommand{\dateseparator}{-} % ISO 8601 date format

% What to display in table of contents
\setcounter{tocdepth}{1}
\setcounter{secnumdepth}{2}

% Lists
\usepackage{enumerate} % Determines the style in which the counter is printed
\usepackage{enumitem} % Provides user control over the layout of the three basic list environments

% Citing & bibliography
\usepackage{csquotes} % For \enquote command for proper quotation marks, also biblatex recommends this
\usepackage[numbers]{natbib}


%%%%%%%%%%% GRAPHICS %%%%%%%%%%%

\usepackage{graphics,color,xcolor}

% Figures
\usepackage{epsfig} % Solves some problems in \includegraphics{<.eps-file>}
\usepackage{graphicx} % More options for \includegraphics
\usepackage{wrapfig} % Figure environment that lets text wrap around figure
\usepackage{float} % Figure placement
\usepackage{caption} % More options for \caption
\usepackage{subcaption} % Subfigures

% Tikz
\usepackage{tikz}
\usepackage{pgf,pgfplots} % Pgfplot
\pgfplotsset{compat=1.15}

% För alduslöv
\usepackage{pifont}


%%%%%%%%%%% PHYSICS %%%%%%%%%%%

% SI units
\usepackage{siunitx}
\DeclareSIUnit\clight{\text{$c$}} % redefine from c_0 to c
\DeclareSIUnit\byte{B}

% Physics macros
\usepackage{physics} % Defines lots of nice commands like \derivative, \norm, \evaluated, etc. It is recommended to use these as much as possible for nice spacing and readable LaTeX code.
\usepackage{braket} % Defines \bra, \ket, \braket, and \set
\usepackage{slashed} % For Feynman slash notation
\usepackage{simpler-wick} % Wick contractions (may require sty-file)
% \usepackage[compat=1.1.0]{tikz-feynman} % Feynman diagrams (has to be compiled with LuaTeX)
\usepackage{tensor} % Covariant index notation


%%%%%%%%%%% CODING %%%%%%%%%%%

% For nice code insertions
\usepackage{listings}
\definecolor{codegreen}{rgb}{0,0.6,0}
\definecolor{codegray}{rgb}{0.5,0.5,0.5}
\definecolor{codepurple}{rgb}{0.58,0,0.82}
\definecolor{backcolour}{rgb}{0.95,0.95,0.92}
\lstdefinestyle{mystyle}{
    backgroundcolor=\color{backcolour},   
    commentstyle=\color{codegreen},
    keywordstyle=\color{magenta},
    numberstyle=\tiny\color{codegray},
    stringstyle=\color{codepurple},
    basicstyle=\ttfamily\footnotesize,
    breakatwhitespace=false,         
    breaklines=true,                 
    captionpos=b,                    
    keepspaces=true,                 
    numbers=left,                    
    numbersep=5pt,                  
    showspaces=false,                
    showstringspaces=false,
    showtabs=false,
    tabsize=4
}
\lstset{style=mystyle}


%%%%%%%%%%% MATHEMATICS %%%%%%%%%%%

% AMS packages
\usepackage{amsmath,amsfonts,amsthm,amssymb}

% Theorem and proof environments
\iflanguage{swedish}{
    \newtheorem{theorem}{Sats}
    \newtheorem*{theorem*}{Sats}
    \newtheorem{proposition}{Proposition}
    \newtheorem*{proposition*}{Proposition}
    \newtheorem{corollary}{Följdsats}[theorem]
    \newtheorem{corollary*}{Följdsats}
    \newtheorem{lemma}{Lemma}
    \newtheorem*{lemma*}{Lemma}
    \theoremstyle{definition}
    \newtheorem{definition}{Definition}
    \newtheorem*{definition*}{Definition}
}{}
\iflanguage{english}{
    \newtheorem{theorem}{Theorem}
    \newtheorem*{theorem*}{Theorem}
    \newtheorem{proposition}{Proposition}
    \newtheorem*{proposition*}{Proposition}
    \newtheorem{corollary}{Corollary}[theorem]
    \newtheorem{corollary*}{Corollary}
    \newtheorem{lemma}{Lemma}
    \newtheorem*{lemma*}{Lemma}
    \theoremstyle{definition}
    \newtheorem{definition}{Definition}
    \newtheorem*{definition*}{Definition}
}{}

% Better version of the \not command
\usepackage{cancel}

 % Does polynomial division for you
\usepackage{polynom}

% Vectors are upright boldface. I think this definition is better than the physics package's \vectorbold.
\let\Vec\undefined % We use \vec w/ lowercase v
\renewcommand*{\vec}[1]{{\boldsymbol{\mathrm{#1}}}}

% Bar, tilde, and hat that scales with what is under them. Basically I just want these to have consistent names
\let\mathbar\overline
\let\mathtilde\widetilde
\let\mathhat\widehat

% Redefine \exp
% Errors occur if this definition is made before some of the packages are loaded
\let\oldexp\exp
\newcommand*{\Exp}[1]{\oldexp{#1}}
\renewcommand{\exp}[1]{\mathrm{e}^{#1}}

% Main number systems
\newcommand{\naturals}{\mathbb{N}}
\newcommand{\integers}{\mathbb{Z}}
\newcommand{\rationals}{\mathbb{Q}}
\newcommand{\reals}{\mathbb{R}}
\newcommand{\compexnumbers}{\mathbb{C}}

% Some of my own shorthands for correct spacing in math environments
\def\divides{\mid} % Proper spacing of vertical bar in division x|y
\def\from{\colon} % Proper spacing of colon in functions f:A→ B
\newlength\mylen % Isomorphic \mapsto
\settowidth\mylen{$\longleftrightarrow$}
\newcommand{\mapsbetween}{\longleftrightarrow\kern - 0.5\mylen\vline height 1.2ex depth -0.0pt\kern0.5\mylen}
\newcommand{\suchthat}{\qq{s.th.}}
\def\definedas{\coloneqq}
\def\defines{\eqqcolon}

\newcommand*{\transpose}[1]{#1^{\!\mathsf{T}}}
\renewcommand*{\complement}[1]{#1^{\mathsf{C}}}
% \newcommand{\conjugate}[1]{\mathbar{#1}}
\newcommand*{\conjugate}[1]{#1^*}
\newcommand*{\hermitianconjugate}[1]{#1^\dag}
\newcommand*{\inverse}[1]{#1^{-1}}

\newcommand*{\closure}[1]{\mathbar{#1}} % Closure of a set
\def\union{\cup}
\newcommand{\Union}{\bigcup\limits}
\def\intersection{\cap}
\newcommand{\Intersection}{\bigcap\limits}

% Lie-groups & algebras, i.g. SU(n)
\newcommand*{\algebra}[2]{\mathfrak{\MakeLowercase{#1}}\left(#2\right)}
\newcommand*{\group}[2]{\mathrm{\MakeUppercase{#1}}\left(#2\right)}


%%%%%%%%%%% MISCELLANEOUS %%%%%%%%%%%

% In-pdf comments through \todo command
\setlength{\marginparwidth}{2cm} % Silence warning about margin size
\usepackage{todonotes}

% Clickable links and refs
\usepackage[hidelinks]{hyperref} 

% Cleverref automatically detects if you are referencing a figure, table, or equation etc
% Cleverref has to be loaded last I think, after babel and hyperref etc
\usepackage[noabbrev, nameinlink]{cleveref}
\crefname{equation}{}{}
\iflanguage{swedish}{ % Tell cleverref to use Oxford comma
	\newcommand{\creflastconjunction}{, och\nobreakspace}
}{}
\iflanguage{english}{
	\newcommand{\creflastconjunction}{, and\nobreakspace}
}{}


% Tag only referenced equations (this is not an ideal package, as it requires etextools, which is buggy and abandoned by its author)
\expandafter\def\csname ver@etex.sty\endcsname{3000/12/31}
\let\globcount\newcount
\usepackage{autonum}

% Intervals on the real line
\let\interval\undefined % To avoid name conflict with etextools
\usepackage{interval}
\intervalconfig{soft open fences}

% For writing \overset{text}&{=} in align environment
\usepackage{aligned-overset} 


\title{Vetenskapshistoria Inlämningsuppgift 6}
\author{}
\date{\today}

% Specifics for TEK130
\usepackage{times}
\linespread{1.5}

\begin{document}
\maketitle


%\begin{itemize}
%	\item Diskutera hur den vetenskapliga kunskapstraditionen historiskt har skiljt sig från den tekniska kunskapstraditionen.
%	\item Beskriv sedan hur ingenjörskåren har fungerat som en brygga mellan teknik och vetenskap,
%	\item och diskutera vilken roll ingenjören historiskt har haft i samhället.
%	\item Reflektera avslutningsvis över jämställdheten på tekniska högskolor. Kvinnor är underrepresenterade på Chalmers, och på många andra tekniska högskolor. Varför är det så? Hur kan det förändras?
%\end{itemize}

Det finns flera vanliga definitioner av \emph{teknik}. Två av dem är följande.  (\cite[s.\ 24--27]{Lundgren2017})
\begin{definition}\label{def:teknik_definition_gammal}
	\emph{Teknik} är metoder för att minska mängden fysisk eller mental ansträngning som krävs för att utföra en uppgift.
\end{definition}
\begin{definition}\label{def:teknik_definition_ny}
	\emph{Teknik} är tillämpningen av vetenskap för att minska mängden fysisk eller mental ansträngning som krävs för att utföra en uppgift.
\end{definition}
\Cref{def:teknik_definition_gammal} är mycket bredare och teknik i den meningen har utövats sedan mänsklighetens början. Tekniken har tidigt i historien i den meningen varit väldigt frånskild vetenskapen då vetenskap inte varit praktisk utan filosofisk. Innan man började lägga fokus på empiri var vetenskap bara naturfilosofi, men under den vetenskapliga revolutionen, i och med den förändrade synen på hur vetenskap bör utövas, utvecklades ett behov av att använda teknik för att undersöka naturen (\cite[kap.\ 2]{Bowler2005}). Ett exempel på hur vetenskapen använt sig av tekniken är Robert Boyles experiment med luftpumpar. Boyle byggde inte dessa själv, utan tog hjälp av specialiserade teknologer som hade kompetensen att bygga dessa invecklade och ömtåliga apparaturer (\cite[s.\ 407--412]{Bowler2005}). Francis Bacon betonade till exempel också vetenskapens användbarhet. Han menade, för att särskilja den nya vetenskapen från den gamla naturfilosofin, att vetenskapens drivkraft var dess praktiska användbarhet (\cite[s.\ 395]{Bowler2005}).

% ### SNACKA MER OM SKILLNADEN MELLAN VETENSKAP \& TEKNIK? ###

Under och efter den vetenskapliga revolutionen utvecklades relationen mellan teknik och vetenskap och de kom att bli allt mer sammanflätade. \Cref{def:teknik_definition_ny} är en mer nutida definition av teknik som understryker denna relation. \Cref{def:teknik_definition_ny} kan också ses implicit anta den \emph{linjära modellen} för teknikens utveckling.
\begin{definition}
	Den \emph{linjära modellen} för teknikens utveckling säger att relationen mellan vetenskap och teknik är hierarkisk. Vetenskapare producerar genom den vetenskapliga metoden nya teorier som teknologer sedan applicerar (\cite[s.\ 391]{Bowler2005}; \cite[s.\ 15]{Lundgren2017}).
\end{definition}
Den linjära modellen har dock kritiserats (\cite[s.\ 18--23]{Lundgren2017}) och om vi antar \cref{def:teknik_definition_gammal} av teknik kan vi hitta ett par illustrerande motexempel.

Robert Hooke var exempelvis en av teknikerna som jobbade på och förbättrade Boyles luftpumpar samtidigt som han utförde egen vetenskap. Han var också senare en framstående naturvetare (\cite[s.\ 397--412]{Bowler2005}). Under den vetenskapliga revolutionen gjorde naturvetare stor skillnad på yrkena vetenskapare och tekniker. I samband med professionaliseringen av vetenskapen ville man höja ribban för vad som krävdes för att kunna kalla sig vetenskapare. Det blev i och med detta svårare att bli erkänd som vetenskapare om man tog en oortodox väg in i vetenskapen (\cite[s.\ 39--50]{Bowler2005}). Hooke hade av denna anledningen svårt att gå från tekniker till att bli naturvetare.% RÖD TRÅD I DET HÄR STYCKET? SNACKA OM ROYAL SOCIETY?

Ett annat tidigt exempel på den ickelinjära relationen mellan vetenskap och teknik är amerikanska jätteföretag som General Electric, AT\&T, och Du Pont. Företagen grundade under början av 1900-talet stora anläggningar för forskning och utveckling, F\&U. Polymerer, transistorer, och Irving Langmuirs arbete för General Electric på att förbättra glödlampan, som gav honom 1932:s Nobelpris i kemi, är exempel på produkter från dessa anläggningar. Det var för företagen uppenbart att de skulle behöva utöva vetenskap parallellt med tekniska innovationer för att förbli konkurrenskraftiga (\cite{Smith1990}).

\bigskip

Författaren Nassim Nicholas Taleb driver i sin bok The Black Swan tesen att mänsklighetens historia till största del formats av enskilda oförutsägbara händelser (så kallade svarta svanar, döpta efter den väldigt oförutsägbara upptäckten av den första svarta svanen gjord av de tidiga kolonisatörerna av Australien) som 9-11, bankkrisen 1982, och uppfinnandet av internet. Taleb menar att detta är sant speciellt i vetenskapens historia. Han tar upp upptäckten av den kosmiska bakgrundsstrålningen som exempel. De två upptäckarna, Arno Penzias och Robert Woodrow Wilson, som senare kom att tilldelas Nobelpriset i fysik för upptäckten, hade inte planerat att leta efter den kosmiska bakgrundsstrålningen. De höll på att kalibrera ett av radioteleskopen på Bell Laboratories i New Jersey, men hade problem med att eliminera ett statiskt brus i bakgrunden. Inte ens efter att de rensat teleskopet på fågelbajs försvann bruset. Det tog väldigt lång tid innan någon av dem kom på att relatera detta till den då väldigt hypotetiska big bangteorin (\cite[kap.\ 11]{Taleb2007}).

Ett annat exempel Taleb tar upp är att ingen på 1900-talets mitt hade kunnat föreställa sig att vi idag använder datorer till att visualisera och upptäcka ny matematik.

Taleb menar med dessa exempel att vetenskap är högst beroende av svarta svanar och att tekniken är ett sätt att exponera sig för sådana (\cite[kap.\ 11]{Taleb2007}). Han upprepar en poäng som redan Francis Bacon kunde sätta fingret på, att de mest betydelsefulla upptäckterna är de mest oförutsägbara. Paradoxen verkar vara att om det hade varit lätt att förutsäga hade det redan varit upptäckt.
%Taleb tar också upp det ikoniska exemplet med Alexander Flemings upptäckt av penicillin.


%\begin{definition}
%	\emph{Serendipitet} är en oavsiktlig upptäckt, en positiv överraskning vid sökandet efter eller utförande av något annat.
%\end{definition}
%
%MEMEA LITE \& SNACKA OM DET ETYMOLOGISKA URSPRUNGET?

\bigskip
% ### SNACKA OM INGENJÖRENS ROLL I SAMHÄLLET ###

Langmuirs Nobelpris, som jag nämnde tidigare i relation till General Electric, är ett väldigt slående exempel på ett fenomen som pågick kontinuerligt under 1800- och 1900-talet. Det blev då ett yrke att kombinera naturvetenskaplig och teknisk kompetens.
\begin{definition}
	En \emph{ingenjör} är någon som tillämpar vetenskap för att lösa tekniska problem. 
\end{definition}
Det fanns från början i princip två skolor om hur ingenjörer skulle bildas, den ena menade att mest fokus skulle läggas på de tekniska tillämpningarna, medan den andra menade att mest fokus skulle läggas på den teoretiska grunden. Den andra skolan har ett snyggt namn, det \emph{polytekniska idealet}, och är den skola som i slutändan segrat och adopterats av flest universitet (\cite[s.\ 243--249]{Sundin2006}). Exempelvis har både KTH, Sveriges mest prestigefyllda tekniska universitet (\cite[s.\ 75]{Berner2000}), och Chalmers en stor del av sin utbildning i teoretiska ämnen. Många olika utbildningar har även samma grundkurser, som linjär algebra och mekanik, som man läser oavsett om man utbildar sig till sjöingenjör eller kemitekniker. %Samt att de kurserna ges av forskare i matematik respektive fysik och inte av ingenjörer.

Yrkesgruppen ingenjörer växte och fick mer prestige under och efter andra världskriget. Ingenjörerna kom att ha samma anseende som de yrkena man vanligtvis förknippar med bildning: vetenskapsmän, läkare, och jurister. Ingenjörerna kom också ofta att inta chefspositioner på företag. Där kunde de tekniska problemen bli mer abstrakta och snarare handla om att effektivisera marknadsföring eller en produktionskedja än att bygga en bro (\cite[s.\ 249]{Sundin2006}). Detta är varför vi har ingenjörsutbildningar som till exempel industriell ekonomi idag.

%\enquote{
%	When the patents ran out and when Theodore Roosevelt decided to enforce the antitrust laws, a number of large companies that had dominated their respective industries established research laboratories as part of a more general reorganization of their operations. The goal was to maintain market share in established product lines by continuous product improvement and overall organizational efficiency.
%} (\cite[s.\ 124]{Smith1990})

%\enquote{
%	Irving Langmuir improved the light bulb dramatically and preserved GE's position in this lucrative market. Langmuir's approach to the light bulb problem was so innovative that it won him a Nobel Prize in chemistry.
%} (\cite[s.\ 126]{Smith1990})

%\enquote{
%	By the late 1930s, however, science was becoming widely regarded as the driving force behind the process of innovation. A shift from leapfrogging to science-based invention had begun in the 1920s, when two of the research pioneers, Du Pont and AT\&T, began to do university-style research in fields of interest to their respective companies. Instead of just applying science, industrial researchers would "do". science. Polymers and solid-state physics were just emerging as disciplines, and Du Pont and Bell researchers made major scientific contributions.
%} (\cite[s.\ 127]{Smith1990})

%\enquote{
%	In the 1950s, industrial research, especially in central research laboratories, moved toward academic style research. In part this reflected the fact that the demand for Ph.D.'s far exceeded the supply, so unhappy researchers could readily find employment in other companies or in academia. Sociological studies on professionalism and the new discipline of research man­agement legitimized the scientists' need for autonomy.
%} (\cite[s.\ 129]{Smith1990}) SNACKA OM DETTA I RELATION TILL INGENJÖRENS ROLL I SAMHÄLLET?



%\enquote{
%	det polytekniska idealet---är teknologins kärna den naturvetenskapliga teorin. Alla teknikens områden har gemensamt teoretiskt, matematiskt
%	innehåll. De tekniska läroanstalternas huvuduppgift skall
%	därför inte vara att lära ut \enquote{know how}, utan att undervisa i
%	naturvetenskaplig teori och tillämpad naturvetenskap.
%} (\cite[s.\ 247]{Sundin2006})


%\enquote{
%	Ett uttryck för teknikernas växande betydelse inom kom-
%	munernas förvaltning var den år 1902 bildade Svenska kommunaltekniska föreningen. Där fastställdes tekniska normer och regler och där hade stadsingenjörer och andra tekniska tjänstemän ett forum för att behandla frågor rörande vatten- och avloppsverk, renhållning, vägar och gator etc
%} (\cite[s.\ 250]{Sundin2006})

%\enquote{
%	Därför har också kommunaltekniken och kommunalteknikerna haft mycket stor betydelse för förändringen av levnadsförhållandena i städer och
%	tätorter från 1800-talets slut.
%} (\cite[s.\ 250]{Sundin2006}) TA UPP DESSA TVÅ I DISKUSSIONEN OM INGENÖRENS ROLL I SAMHÄLLET.


% ### SNACKA OM JÄMSTÄLLDHET ###

Ingenjör har genom historien varit ett väldigt mansdominerat yrke. I Sverige dröjde det till 1921 innan kvinnor officiellt antogs till något av dom tekniska universiteten. Detta är väldigt sent om man jämför det med att kvinnor redan på 1870-talet antogs till vetenskaperna kemi, matematik, och fysik (\cite[s.\ 78--80]{Berner2000}). En av anledningarna till detta skulle kunna vara, som tidigare nämnt, dels att ingenjör blev ett väldigt ansenligt yrke och dels att ingenjörerna ofta intog chefspositioner. Att leda och kunna ge order till andra (män) ansågs vara något endast män borde göra (\cite[s.\ 85--90]{Berner2000}). En annan anledning skulle kunna ha varit att det under 1800-talet, i samband med utvecklingen av biologi, vuxit fram en bild av kvinnan som skör och öm. I ett civiliserat samhälle borde inte en kvinna utföra påfrestande uppgifter. Det ansågs äventyra hennes femininitet (\cite[s.\ 83--89]{Larsson2006}). 1893 gav KTH avslag på en ansökan från en kvinna med dels just den motivationen, att den typen av intensiva studier som bedrevs på KTH inte passade för medelklasskvinnan.

Datoringenjör och programmerare är speciellt mansdominerade genrer av ingenjör. Detta är dock, kanske förvånande, en relativt ny inställning till programmering. Datorindustrin var till en början öppen för kvinnor. Detta berodde till stor del på att tidiga datorer inte var lika komplicerade och abstraherade som senare datorer. Arbetsuppgifterna för en tidig programmerare var mekaniska och repetitiva, snarare än tekniska och komplicerade. Kvinnor kunde då utnyttjas som billig arbetskraft (\cite[s.\ 43--45]{Ensmenger2015}).

Övergången av programmerare från ett kvinnligt yrke till ett manligt yrke skedde successivt och av flera anledningar. En av anledningarna var att behovet av programmerare exploderade under 1960-talet och därför både höjdes medellönen samtidigt som antalet programmerare ökade. Då kunde välutbildade män från angränsande områden som matematik och elingenjörer lätt göra övergången till programmering (\cite[s.\ 43--50]{Ensmenger2015}). En annan anledning skulle kunna vara att datorer var stora och dyra maskiner som privatpersoner inte hade råd med. De som kunde komma åt datorer att börja programmera på var ofta universitetsanställda, vilket också var ett mansdominerat yrke (\cite[s.\ 43--45]{Ensmenger2015}). En tredje anledning, och kanske den mest sannolika, är att arbetsgivare diskriminerade. I och med att yrket blev mer seriöst och mer betydande ville arbetsgivare anställa mer seriösa och kompetenta anställda. I sållandet bland sökande till en tjänst hade män en fördel, de sågs automatiskt som mer seriösa (\cite[s.\ 50--51]{Ensmenger2015}).

Resultatet av den här maskuliniseringen av många ingenjörsyrken har blivit att det kvinnor fortfarande är underrepresenterade på de tekniska högskolorna, även om de formella barriärerna är borta sedan länge. Många högskolor jobbar aktivt med att jämna ut fördelningen över könen. Till exempel har Chalmers investerat \SI{300}{\mega SEK} i en tioårsplan, GENIE, för att öka andelen kvinnor i fakulteten och överlag underlätta för kvinnor att bygga en karriär på Chalmers (\cite{genie}). Även mindre konkreta saker görs för att underlätta för kvinnor att söka sig till tekniska högskolor, som erkännandet av kvinnliga vetenskapares bidrag till vetenskapen. Lise Meitnerdagarna på Chalmers är döpta efter kärnfysikern Lise Meitner som 1944 nekades ett Nobelpris för sina upptäckter om kärnklyvning.

Sammanfattningsvis ser vi att det finns ett intresse av att göra ingenjörsutbildningarna mer välkomnande för kvinnor. Vi har även fått en inblick i varför detta historiskt sett inte varit fallet.







\appendix
\printbibliography

\end{document}