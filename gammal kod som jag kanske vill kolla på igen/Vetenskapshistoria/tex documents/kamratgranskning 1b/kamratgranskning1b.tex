\documentclass[a4paper, 11pt]{article}
\usepackage[utf8]{inputenc}
\usepackage[swedish]{babel}

\title{Vetenskapshistoria TEK130\\ Kamratgranskning I}
\author{}
\date{\today}

\begin{document}
\maketitle


I din text har du redogjort för de sätt att tänka kring biologi som växte fram under 1800-talet. Du har även tagit upp den förändrade synen på sexualitet och kön samt diskuterat hur kvinnors möjligheter att delta i naturvetenskaplig forskning förändrats under 1800- och 1900-talet. Du använder rubriker för att visa vilken av frågeställningarna du svarar på med vilket stycke, vilket jag tycker är tydligt.

Du har kopierat mycket av en mening från källan Larsson 2006, vilket jag inte tycker är jättesnyggt. Jämför
\begin{quote}
	Läkare på 1800-talet behandlade borgarklassens reproduktiva hälsa som en statsangelägenhet och frågan om hur kvinnor skötte sina kroppar var intimt förknippad med föreställningar om hela nationens styrka och livskraft.
\end{quote}
som det står i din text, med
\begin{quote}
	Läkarna behandlade borgarklassens reproduktiva hälsa som en statsangelägenhet och frågan om hur kvinnor levde sina liv och skötte sina kroppar var intimt förknippad med föreställningar om hela nationens styrka och livskraft.
\end{quote}
som det står i Larsson 2006.

Annars tycker jag din text är välskriven, speciellt delen om kvinnors möjlighet att delta i naturvetenskapen under 1800-och 1900-talet, även om det är lite otydligt vem som menar att ''people were not what they were born but what their education hade made them'' eller vad det betyder.


%
%Texten är överlag lätt att läsa. Finns några meningar som inte är helt sammanhängande. Det handlar säkert bara om att läsa igenom texten en gång till efter att den är klar. Du har också använt ordet ''så'' på tre olika ställen. På varje ställe hade det kunnat skippas.
%
%Ett förbättringsförslag hade kanske kunnat vara att binda ihop texten mer, till exempel mellan stycke tre och stycke fyra kunde jag inte följa någon röd tråd. Det hade också kanske varit intressant att ställa det sista påståendet i texten, ''mycket byggde på fördomar och att underliggande åsikter drev vilka slutsatserna var'', i kontrast till den nya vetenskapens fokus på experiment och syn på sig själv som antidogmatisk.






\end{document}