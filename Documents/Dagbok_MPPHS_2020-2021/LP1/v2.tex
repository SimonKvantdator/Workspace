%! TEX root = /home/simon/Documents/Dagbok_MPPHS_2020-2021/main.tex
\subsection{Måndag 2020-09-07}

Rövning i integrationsteorin idag \& sen lunch med Arvid. Vi snackade lite om att han hittar på mycket grejor. Han sa att han inte gillar att planera så mycket, \& kommenterade på att jag brukar skriva ner saker i min kalender. Jag känner nog att jag har behov av att hitta på saker mer regelbundet nu. Det känns som att jag inte gjort det så mycket sedan sommaren tog slut. Speciellt lite mer spontana saker. jag ska se om jag kan hänga med morsan hem idag eller någon annan av dagarna i veckan \& kanske träffa med Lukas \& Linnéa lite.

Var på rövning i kommutativ algebran \&så idag. Det verkar vara rätt high value faktiskt. Han bevisar satser \& sånt han vill använda. Tror jag vill försöka skumma igenom mina egna lösningar innan varje rövning i fortsättningen.
Kolla igenom vad han skrev om uppg 0.7 (hint om att det kan komma på tentan)?

Snackade med Elin lite om det här med att hitta på saker mer regelbundet. Hon sa att hon har ett större behov av att få ordning på matlagning \& sånt först. Vi snackade \&så om att hon gärna hade haft en dags förvaning om jag ska planera något med henne, men att jag inte behöver varna i förväg om jag ska iväg själv med typ Arvid eller Gurra.


\subsection{Tisdag 2020-09-08}

Vi definerar ett nytt kommando,
\begin{align}
	\set{1, 2, 3} \mapsbetween \set{2, 3, 4}.
\end{align}

Jag kanske vill lägga tisdgsmorgnar på att läsa igenom materialet för kommande kommutativ algebra-föreläsning?

TODO: läs proposition 1.5 i Reid igen \& förstå beviset (Per rekommenderar).

\begin{align}
	\mathbar{A_A}
\end{align}

Jag har överlag varit rätt dålig på att hålla fokus dom senaste dagarna. Kanske är för att det inte finns så mycket konkret att göra. Men nu har iaf standardmodell-läxan dykt upp. Arvid \& jag ska sätta oss med den ngn av dagarna.


\subsection{Onsdag 2020-09-09}

Idag vill jag börja kolla på läxan i standardmodellen. Kanske läsa igen läxan så jag vet den handlar om \& sen kolla på lite av ferrettis youtubevideor. I eftermiddag vill jag även skumma igenom elektros matlab.

TODO: skapa ett eget paket med mina mattekommandon? Idéen kan vara att det extendar physics-paketet.

Main-kursboken i standardmodellen är \enquote{Standard Model: An Introduction} by S.F. Novaes. Kanske ska försöka sitta \& läsa i den snarare än att titta på hans videos. Det blir så passivt att sitta \& kolla på videos liksom. TODO: läs 1.2, 1.3, \& 1.4.1 i Novaes.

\bigskip

Hittade en najjs lista på böcker som Scientific American tyckte \enquote{shaped a Century of Science}:
\begin{itemize}
	\item The principles of quantum mechanics (1930) by Paul Dirac
	\item The collected papers of Albert Einstein, volume 2 : the Swiss years : writings, 1902-09 by Albert Einstein
	\item Fractals : form, chance, \& dimension (1977) by Benoit B. Mandelbrot
	\item Nature of the chemical bond and the structure of molecules and crystals; an introduction to modern structural chemistry (1939) by Linus Pauling
	\item Principia mathematica (1910-13, 3 vols. Vol. 1; Vol. 2; Vol. 3) by Bertrand Russell and Alfred North Whitehead
	\item A search for structure : selected essays on science, art, and history (1981) by Cyril Smith
	\item Theory of games and economic behavior (1944) by John von Neumann and Oskar Morgenstern
	\item Cybernetics : or, Control and communication in the animal and the machine (1948) by Norbert Wiener
	\item The conservation of orbital symmetry (1970) by R. B. Woodward and Roald Hoffmann
	\item The meaning of relativity (1922) by Albert Einstein
	\item QED : the strange theory of light and matter (1985) by Richard Feynman
	\item The art of computer programming (1968) by Donald Knuth
\end{itemize}


\subsection{Torsdag 2020-09-10}

Vi testar lite nya kommandon:
\begin{align}
	\limsup_{i \to \infty} a_i \definedas \lim_{i \to \infty} \sup \set{a_j \suchthat j \geq i}\\
	\lim_{i \to \infty} \sup \set{a_j \suchthat j \geq i} \defines \limsup_{i \to \infty} a_i
\end{align}

TODO: skriv till Jeffrey \& säg att du hellre vill att han kör sina powerpoints.

Hade varit ett lagom långt projekt att installera ngn lightweight linux på ett USB för att typ kunna ha med sig python, C, vim, latex, etc överallt. \enquote{TIP: Use 32 bit Linux OS to make it compatible with any available PC.} ven kanske en najjs idé att ta ett relativt stort USB (\SI{16}{\giga \byte}) \& skapa två partitioner, en med lubuntu \& en som ett vanligt USB.

\bigskip

Hamburgare i Wijkanders idag, tagga!

Ska dra hem med morsan sen \& hälsa på katterna lite. \& sen hnga med Lukas \& Linnea.

\bigskip

Kanske behöver ta Williams rövning på Tisdag 15/9. Han vet på måndag om han har labb då eller inte. Vi bestämde att jag gör det. William ska skicka sina lösningar.

\bigskip

Fick massa problem när jag skrev \verb|\let\definedAs\coloneqq| förut så jag fick skriva \\\verb|\def\definedAs{coloneqq}| istället. Vet ej vrf det fuckade men orkar inte undersöka just nu.


\subsection{Fredag 2020-09-11}

Var på Davids kontor förut \& påminnde honom om exjobbet. Han verkade fett stressad överlag.

\bigskip

Hittar inte mitt \SI{32}{\giga \byte}-USB, det kanske ligger hemma hos mina päron. Hade tänkt partitionera upp det i \SI{17}{\giga \byte} minne, \SI{12}{\giga \byte} lubuntu med mina egna appar \& inställningar (hela lubuntu tar upp $\sim \SI{940}{\mega \byte}$), \& \SI{4}{\giga \byte} ubuntu startup disk.

En lit idé hade varit att se om det går att köra lubuntu från mitt xbox360.

TODO: fixa index i standardmodelläxan 1.(iii). \checkmark


\subsection{Söndag 2020-09-13}

Satt på Chalmers idag med standardmodellen. Kanske borde fråga Ferretti vad en spinor är? Det känns som att uppgifterna inte är supersvåra men lite svåra att kontextualisera.

Eric var på Chalmers \&så \& det var rätt mysigt att sitta \& plugga med honom, han sa att jag borde dricka äggula på morgonen bland annat :).


