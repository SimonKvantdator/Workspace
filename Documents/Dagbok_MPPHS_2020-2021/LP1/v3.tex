%! TEX root = /home/simon/Documents/Dagbok_MPPHS_2020-2021/main.tex
\subsection{Måndag 2020-09-14}

Vill sitta mycket med standardmodellen den här veckan. Kanske sikta på att göra alla tre uppgifterna.

\bigskip

Vi rövningsledare bestämde att folk får maila in sina laborationer istället för att redovisa dem på labtillfällena. På så sätt får vi mer tid åt att hjälpa folk.

\bigskip

Undra om det hade gått att anteckna på datorn med vim om jag hade haft tillräckligt bra fingersättning.

Satt typ en timme med Vim för råkade uppdatera \& i någon ny version fanns någon fett jobbig \enquote{parentesmatchare}. Men fick äntligen ordning på det. \href{https://vimrc-dissection.blogspot.com/2006/08/vim-turn-that-showmatch-crap.html}{\color{blue}Här} är en diskussion som speglar mina känslor just nu. Lösningen var att köra\\ \verb|sudo find / -name "matchparen.vim" -exec rm -f {} \;|.

\subsection{Tisdag 2020-09-15}

TODO: lägg upp ett monorepo på github med all min kod?
TODO: lägg in Williams \& dina lösningar i Overleaf. \checkmark

TODO: rätta labbar. \checkmark


\subsection{Onsdag 2020-09-16}

Lämnade in homework 1 standardmodellen idag. Var lite grejor jag inte helt förstod så mailade Ferretti \& han sa att vi kunde ta ett möte på fredag eftermiddag \& snacka lite om det. Hade tänkt fråga om
\begin{itemize}
    \item Hur verkar charge conjugation på derivator?
    \item I Peskin \& Schroeder finns en formel på hur charge conjugation verkar på spinorer, men den innehåller ett transponat, \& eftersom transponat är basberoende tyckte jag det var rätt konstigt. I min uträkning får jag upp massa konstiga saker på grund av det, som typ konjugat av $\partial_\mu$ och konjugat av spinorer.
    \item Vad är en spinor? Vad händer när jag tar konjugat av en spinor?
\end{itemize}


\subsection{Torsdag 2020-09-17}

Ska springa med Kevin \& Lindgren idag :)

\bigskip

Ska även på middag med Päronen imorgon, så får Elin träffa Petter \& Hanna.

\bigskip

Vi ska övervaka en dugga sen idag på rövningen. Känns inte superuppstyrt men vi får se hur det går.

\bigskip

TODO: gör Vimtutor igen.


\subsection{Fredag 2020-09-18}

Hade möte idag med Ferretti. Vi snackade i typ en timme om spinorer. Basically sa han att en spinor är definerad utifrån en rep. Så alla saker jag såg som jag tyckte var basberoende var rätt normala. Den vi kollade på i inlämningen var en irrep-uppdelning av en $\compexnumbers^4$-rep (av Lorentzalgebran?) i två $\compexnumbers^2$-reps. Det var fett bror av honom dock att ta ett möte i en hel timme bara för att förklara spinorer för mig.

Idag på mötet med rövningsledarna \& Ville bestämde vi att vi bara ska räkna demouppgifter på tisdagar \& sen bara hjälpa folk på torsdagar. Vi snackade även om extrapass som vi hade kunnat hålla bara för dom eleverna som känner behovet. Vi sa även att jag ska höra med den eleven i phaddergrupp B som snackade med mig under matlabpasset \& frågar om hon känner till fler som har det behovet.

\bigskip

Okej, ny konvention för \LaTeX-kommandon: små bokstäver, e.g.\ \verb|\suchthat|.


\subsection{Lördag 2020-09-19}

Vi tog en rätt seg dag idag \& tog en promme med Hampus i skogen innan vi drog hem till Göteblorg igen.

\bigskip

Jag installerade Ubuntu 20.04 på mitt \SI{32}{\giga \byte} usb-minne, $A$, idag. Jag började med att byta partition table till GPT. Sen partitionerade jag det i \SI{16}{\giga \byte} ext4, \SI{250}{\mega \byte} FAT32, \& resten oallokerat. Sen bytte jag flagga på FAT32-partitionen till \emph{boot} \& \emph{esp}. Idén var att jag ville skapa en \emph{EFI System Partition} som UEFI kan boota från.

Nästa steg var att ladda ner .iso-filen \& göra en startup disk på ett annat usb-minne, $B$, med Startup Disk Creator. Sen bootade jag $B$ från en av mina gamla datorer (för att inte paja ngt med booten på min main) \& körde programmet Install Ubuntu 20.04 LTS. Jag valde keyboard layout, etc. Sen valde jag \emph{minimal installation}, sen \emph{something else}, sen valde jag ext4-partitionen på $A$ \& valde hela $A$ som \emph{Device for boot loader installation}. Sen \emph{Install now}.

Innan jag fick pail på detta råkade jag dock nästan bricka pappas dator. Jag körde ett installationsusb med Lubuntu \& jag antar att den skrev över den boot manager som fanns där redan, för fick meddelandet \enquote{Unable to find operating system}. Det var väldigt dumt att köra på pappas dator. Men han sa att han inte hade haft något viktigt så jag bara installerade Ubuntu igen åt honom. Så jag antar att det inte var något fara. Men väldigt dålig idé ändock.

Jag kommer inte in i BIOSen på min vanliga dator. Detta e lite småjobbigt, men kanske inte ett superproblem just nu. Jag läste att man skulle kunna återställa BIOSen genom att dra ut ngt RC-batteri eller kortsluta ngn grej i datorn. Det lät risky i varje fall så har inte superbråttom med detta.

\bigskip

Pajade skärmen på min mobil \&så. Det var lite RIP men tror det går fine att ersätta. Egentligen, nästa gång jag byter telefon, borde jag köra på konceptet att jag köper en gammal teflon billigt \& installerar /e/ eller ngt annat linux.

\subsection{Söndag 2020-09-20}

Visade sig att jag bara kunde köra ubuntu från USB-stickan om den var kopplad till datorn där jag installerade på. Detta e fett jobbigt men jag hittade \href{https://meaningofstuff.blogspot.com/2019/09/linux-ubuntu-1904-full-install-on-usb.html}{\color{blue}en guide} som gick igenom hur man kan kopiera över booten till USB \&så. Ska följa den när jag har tid. 
