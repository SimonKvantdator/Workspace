%! TEX root = /home/simon/Documents/Dagbok_MPPHS_2020-2021/main.tex
\subsection{Måndag 2020-10-05}

Ska springa med Kenneth sen idag. Tänker att jag visar honom Safjället \& sen kanske lite utegym.

Han hade dock en deadline så vi sköt upp till onsdag. Men jag gjorde ett event av det i min kalender som jag kunde dela med honom, så nu kommer han inte undan. Safjället any\% speedrun onsdag 7/10 19:00.

\bigskip

Man kan skriva
\begin{verbatim}
	\usepackage{xcolor}
	\pagecolor[rgb]{0,0,0}
	\color[rgb]{0.5,0.5,0.5}
\end{verbatim}
för att få darkmode i sin pdf.

\verb|Tweepy| är en python-wrapper för twitters API, ska ju även finnas någon najjs twitterklient som man kan köra från terminalen :).\verb|Pytrends| är en python-wrapper för googles API.

\bigskip

Robert Berman svarade \& sa att han redan handledde två exjobb. Mailade Bäckdahl igen, ska försöka få till ett möte med honom.

Bäckdahl svarade \& verkade fortfarande taggad så ska ha ett möte med honom onsdag nästa vecka. Tror detta kommer bli bra!

Martin Raum från HPC:n svarade nu \&så \& tyckte inte att det skulle vara några problem att läsa kursen i efterhand. Han gav mig lite tips till \& med :). Jag ska maila Bengt-Erik Mellander för att regga mig på HPC:n snarast.

\bigskip

Vill kanske fortsätta läsa om Hardy-Littlewoods funktion imorgon. Sen kanske maila Jeffrey om \enquote{fixed uniform proportion} på s.\ 92.


\subsection{Tisdag 2020-10-06}

Mailade Bengt-Erik nu \& frågade om han ville anmäla mig till HPC:n. Insåg \&så att jag inte är anmäld till sannolikhetsteorins grunder. Det kanske är lite RIP. Borde typ höra med Bäckdahl vilka kurser han vill att jag ska läsa.

Bengt-Erik sa att kursen var full men att jag kunde maila Joakim Norbeck. Så jag mailade Norbeck \& han reggade mig på kursen :).

Thomas Wernståhl mailade om uppdrag LP2 men jag tror inte jag är super taggad på att ta på mig mer nästa LP asso.


\subsection{Onsdag 2020-10-07}

Fick rätt mycket integrationsteori läst igår, kände mig fett produktiv. Men kanske ska kolla lite på standardmodellen idag då. Kan börja med att kolla igenom hw3 som jag fick tillbaka. \checkmark

Vill kanske börja skriva ner lite bevis i integrationsteorin efter lunch. Kolla om det finns en bevislista. Eventuellt om jag vill kolla på lite subatomärtentor sen.

Det finns nu en bevislista till integrationsteorin,\\
Longer (harder) proofs:
\begin{enumerate}
	\item Theorem 3.20 (Assuming Caratheodory's Theorem 3.25)
	\item Caratheodory's Theorem 3.25
	\item Theorem 3.26 (You do not need to know the proof of Dynkins pi lambda Theorem  although of course you need to know and understand the statement)
	\item Theorem 3.27
	\item Proposition 4.6 
	\item Theorem 4.15
	\item Theorem 5.5 for finite measures (This includes proving Prop. 5.4(i) but not (ii))  (Theorem 5.1 is assumed here.) (Of course by a finite measure, I don't mean that X is finite but only that m(X) is finite.)
	\item Theorem 6.7
	\item Theorem 7.8 for finite measures.
	\item  Theorem 7.9 for finite measures.
	\item  Theorem 8.9
	\item  Theorem 8.10
	\item  Proposition 9.5
\end{enumerate} 
Shorter (Easier) proofs:
\begin{enumerate}
	\item Theorem 3.14
	\item Proposition 3.30
	\item Lemma 3.35
	\item Proposition 4.7 AND 4.8
	\item Corollary 4.16 AND 4.17
	\item Fatou's lemma (assuming the MCT)
	\item The lebesgue dominated convergence Theorem  (assuming Fatou's lemma)
	\item Proposition 4.25
	\item Theorems 4.27 AND 4.28
	\item  Theorem 6.7(i) Only the WLLN
	\item  Proposition 7.10
	\item  Theorem 8.6
\end{enumerate}

\bigskip

Jag skrev till Carlanderska om tån igen.


\subsection{Torsdag 2020-10-08}

Jag hittade \href{http://www.cs.toronto.edu/~tijmen/gnumpyTr.pdf}{\color{blue}en artikel} som beskriver hur man använder gnumpy. Det verkar fett användbart.


\subsection{Fredag 2020-10-09}

Subomtenta idag. Har ej pluggat ngt sedan förra omtentan men tänker att sannolikheten att jag klarar denna tentan är betydligt högre om jag tar den än om jag inte tar den.

Drar till Elins päron ikväll.
