%! TEX root = /home/simon/Documents/Dagbok_MPPHS_2020-2021/main.tex
\subsection{Måndag 2020-09-28}

Lade till NERDCommenter-pluginet till mitt neovim. Man använder det genom att skriva
\begin{itemize}
	\item \verb|,cc| för att kommentera en/flera rader
	\item \verb|,cu| för att avkommentera en/flera rader.
\end{itemize}

Tänkte försöka skapa ett \enquote{monorepo} med all min kod, som Lindgren. Det blev dock lite whack för jag tänkte skriva en \verb|.gitignore| där jag började med att exkludera allt \& sen inkludera saker efterhand, som en \verb|.gitinclude| liksom, men det verkar som att jag inte kan exkludera filer igen efter det. Om jag skriver\\
\begin{verbatim}
file.txt
!file.txt
file.txt
\end{verbatim}
i min \verb|.gitignore| blir alltså \verb|file.txt| ändå medtaget i repot.

Jag lade alldeles för mycket tid idag på att försöka lösa detta med regex i python. Borde inte lägga mer tid på det.

Asso måndagar överlag brukar mest vara rövningar. Jag kanske ska försöka skippa att vara på föreläsningar på måndagar \& istället bara läsa i boken ellr ngt. Tror det hade varit mer produktivt.

Imorgon ska jag räkna lite uppgifter att lägga upp i \enquote{Renskrivna Demouppgifter} på Canvas. \checkmark


\subsection{Tisdag 2020-09-29}

Inte så strukturerad dag idag, menmen, fick lite kommutativ algebra läst. Fick \&så gjort dom flesta av veckans demouppgifter i TMV157. Ska hålla rövningen på torsdag. Tagga försöka förklara vrf termerna i $\derivative[n]{}{x}f(x) g(x)$ blir Pascals triangel!

Imorgon vill jag nog bara sitta med hw3. Får väl fråga typ Eric Nilsson eller ngn vilka videor som e relevanta om jag inte kan lista ut det sälv.

\bigskip

Fick även ett mail imorse om att systembolaget inte kommer ge ut produktinfo längre via sin API, utan bara butiksinfo. Detta är fett RIP för nu kan Arvid \& jag ju inte på riktigt realisera vår \enquote{efficiently exploring systembolaget}.

\subsection{Onsdag 2020-09-30}

Jag kan SSHa in på min dator även om den sover :) high value info.

\bigskip

TODO: kolla upp de tdär som Elin snackade om om att Chalmers hade samarbete med ngn vårdcentral dit jag skulle kunna gå \& få tips på hur jag ska kunna plugga koncentrerat.

TODO: lämna in arvodesräkning \& deltagarlista efter matlablaben i eftermiddag. \checkmark

Det bästa sättet att signera en PDF i linux är att
\begin{enumerate}
	\item öppna den i Xournal
	\item signera med ritverktyget
	\item File $\blacktriangleright$ Export to PDF
\end{enumerate}

\bigskip

Jag inser nu att jag borde kanske läst HPC istället för standardmodellen... Vet ej om jag vill läsa strängteorin nästa LP heller, men då finns det bara kursen Advanced Simulation and Machine Learning som jag skulle kunna läsa. Hade typ velat göra HPC:n nästa LP. Undra hur bedömningen ser ut i HPC:n, om jag kan läsa den nästa LP \& få den rättad innan året är slut? Antagligen inte.


\subsection{Söndag 2020-10-04}

\begin{itemize}
	\item \verb|zo| för att öppna en section, eller \enquote{fold}, i vim
	\item \verb|zc| för att stänga en section.
\end{itemize}

Jag mailade Martin Raum som håller i HPC:n \& frågade om jag kan göra ursen nästa LP.



