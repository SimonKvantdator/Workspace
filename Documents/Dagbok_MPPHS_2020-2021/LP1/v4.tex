%! TEX root = /home/simon/Documents/Dagbok_MPPHS_2020-2021/main.tex
\subsection{Måndag 2020-09-21}

Diskuterade min idé om att använda epsilon-greedy för att maximera njutningen av vin med Arvid idag. Han verkade fett taggad!

Systembolaget hade en API, \url{https://api-portal.systembolaget.se/}

TODO: Lägg upp \verb|get_probabilities_from_reviews.py| i github. For now kan vi köra på att man manuellt skriver in antal av varje betyg i .json-filen. \checkmark

Jag satt till 22:15 med den här idén. Men nu har jag lyckats hämta all data snyggt från Systembolaget iallafall! Fick dock väldigt lite standardmodell gjort.

Arvids farsa sa att han ville vara försökskanin. Det här kommer bli lit!

\bigskip

TODO: fixa en bättre prior till \verb|get_probabilities_from_reviewspy| genom att hitta ngt najjs API till typ google reviews eller yelp eller ngt?


\subsection{Tisdag 2020-09-22}

Satt med Arvid idag lite efter föreläsningarna \& snackade om hw2. Det går inte supersnabbt framåt alltså. Så jag satt inte kvar superlänge utan tänkte att ag skulle sticka hem \& göra ngt annat ist. Typ laga Halloumi Stroganoff.

Vill sitta med hw2 imorgon hela dagen i princip. Men vill ÄVEN ha läst integrationsteorin i förväg inför iövermorgon. Dessa är rimliga mål inför imorgon.


\subsection{Onsdag 2020-09-23}

Var inte supereffektiv idag men fick iaf läst lite \& gjort en uppgift i integrationsteori, gjort (typ) klart Q3 i sm, \& hållt i en matlabövning.

Vi beslutade idag med Ville att vi bara kör demouppgifter på torsdagar.

\bigskip

Det visade sig att lubuntu installerat sig helt korrekt på usb-stickan. Så jag kunde köra den från båda mina datorer. Det var ballt! Jag installared neovim \& lite pythonpaket :)

\bigskip

Elin \& jag pratade idag. Vi kanske ska försöka byta psykiatriker som hon går till.


\subsection{Torsdag 2020-09-24}

Jag testade att göra en startup disk av mitt \SI{4}{\giga \byte}-usb \& sen bara kopiera över det med GParted till mitt \SI{32}{\giga \byte}-usb. Det fungerade att boota sen! Så nu har jag ett \SI{32}{\giga \byte}-usb med
\begin{enumerate}
	\item en \SI{16}{\giga \byte} (persistent) Lubuntu 18.04-installation (med \SI{4}{\giga \byte} swap)
	\item en \SI{4}{\giga \byte} Ubuntu 20.04 iso för att testa \& installera Ubuntu 20.04
	\item \SI{8}{\giga \byte} vanligt minne.
\end{enumerate}

Hittade lite najjs shortcuts i firefox btw,
\begin{itemize}
	\item \verb|Ctrl + T| för ny tab
	\item \verb|Ctrl + Tab| för att bläddra bland tabbar (finns en inställning \enquote{\texttt{Ctrl + Tab} cycles through tabs in recently used order} som man inte vill ha ibockad) % I can apparently not use \verb in a quote..
	\item \verb|Ctrl + K| för att hamna i sökfältet
\end{itemize}

Kevin sade ngt fett intressant idag när vi var ute \& sprang. Tydligen är QWERTY designat på 1800-talet \& inte alls är optimalt för att skriva på. Det är optimerat för att undvika att tangenter fastnade i varandra som dom ofta gjorde på gamla skrivmaskiner om man använde tangenter precis bredvid varandra. Jag har alltså blivit grundlurad \& spenderat mer än en månad på att lära mig korrekt fingersättning i QWERTY. Varför började jag inte med DVORAK ellr ngt om jag ändå skulle göra det ordentligt{\textinterrobang} Aja, nu är det försent.

Vi snackade även om att käka lunch på ngt najjs ställe imorgon. Men kanske drar ngn stans med Elin istället iom att hon kommer hem imorgon förmiddag.


\subsection{Fredag 2020-09-25}

Idag vill jag
\begin{enumerate}
	\item kolla på lite föreläsningar på förmiddagen
	\item göra klart hw2 i sm
	\item komplettera hw1 i sm.
\end{enumerate}

Kanske vill störa David Witt lite \&så efter lunch om jag får chansen. \& kanske börja fila på ett mail till Bäckdahl.

\bigskip

Jag lade till scriptet LargeFile i vim som disablear lite saker om man öppnar stora filer för att det inte ska bli för tungt. Jag lade till
\begin{verbatim}
Plug 'vim-scripts/LargeFile'
\end{verbatim}
och
\begin{verbatim}
let g:LargeFile = 100 " How large (in MB) files should I use the LargeFile
script for?
\end{verbatim}
i min \verb|init.vim|.

Jag tror att jag i framtiden kommer kunna hitta en del najjs scripts på den \href{https://github.com/vim-scripts}{\color{blue}här} githuben.


\subsection{Lördag 2020-09-26}

Tog hand om Elin idag. Jag tror det kommer bli bra. Vi får hjälp. Hon ska träffa sin psykiatriker på måndag och en psykolog på tisdag.


\subsection{Söndag 2020-09-27}

Lämnade Elin med Kenneth några timmar medan jag åkte \& lämnade tillbaka bilen. Det är lugnare nu.

