%! TEX root = /home/simon/Documents/masterarbete/kontinuerlig\ tes/main.tex
\chapter{Mathematical preliminaries}%
\label{cha:mathematical_preliminaries}


\section{Tensors}%
\label{sec:tensors}

% Wald p. 57: Multilinear maps on are a very general class of entities Let us consider, now, quantities of physical interest in space. All experiments in physics measure numbers, so all quantities of physical interest must eventually be reducible to numbers. However, many quantities of interest—such as the magnetic field or the stress tensor mentioned at the beginning of section 2.3—require the additional specification of a basis of vectors in order to produce numbers. A very general class of quantities of interest are maps of vectors and dual vectors into numbers. Since any such analytic map can be Taylor-expanded as a sum of multilinear maps, we see that tensor fields—i.e., multilinear maps of vectors and dual vectors into numbers—encompass an extremely wide class of quantities.

A very general class of physical quantities is vectors. Since the outcomes of experiments are numbers, or \emph{scalars}, a very general class of maps are \emph{multilinear maps}, mapping sets of vectors to scalars. Tensor products naturally identify multilinear maps defined on Cartesian product spaces with linear maps defined on tensor product spaces. The following definition captures this in a universal property. 

\begin{definition}\label{def:tensor}
	Let $R$ be a commutative ring with unity, we shall use $R = \complexnumbers$. Let $U$, $V$, and $T$ be $R$-modules and let $\pi \from U \times V \to T$ be bilinear. Then $\pi$, or $T$, has the \emph{universal property for tensor products} if, for any $R$-module $W$ and $R$-bilinear map $f \from U \times V \to W$, there exists a unique $R$-linear map $g \from T \to W$ such that $f = g \circ \pi$.
\end{definition}
If two $R$-modules both have the universal property for tensor products, then it follows that they are isomorphic.\todo{cite Hjalmar?}

It can also be shown that $U \otimes V$, defined as $F(U \times V) / \!\sim$ with $F(U \times V)$ being the free module generated by $U \times V$ and $\sim$ being the equivalence relation $(u, v) + (u', v) \sim (u + u', v),~ r (u, v) \sim (r u, v) \sim (u, r v)$, indeed has the universal property for tensor products. Hence the class of \emph{tensors} (maps like $g$) is a realization of a very general class of maps of physical quantities as linear maps.


\section{Spinors}%
\label{sec:spinors}

\subsection{What are spinors?}%
\label{sub:what_are_spinors_}

Let $W$ be a two-dimensional vector space over $\complexnumbers$. Then we may define the \emph{dual space} $\dual{W}$ as the space of linear maps from $W$ to $\complexnumbers$, the \emph{conjugate dual space} $\dual{\conjugate{W}}$ as the space of antilinear maps from $W$ to $\complexnumbers$, and the \emph{conjugate space} $\conjugate{W}$ as $\dual{(\dual{\conjugate{W}})}$. A \emph{tensor of type $(k, l; k', l')$} is a linear map from
\begin{align}
	\underbrace{\dual{W} \otimes \dots \otimes \dual{W}}_{\times k}
	\otimes \underbrace{W \otimes \dots \otimes W}_{\times l}
	\otimes \underbrace{\dual{\conjugate{W}} \otimes \dots \otimes \dual{\conjugate{W}}}_{\times k'}
	\otimes \underbrace{\conjugate{W} \otimes \dots \otimes \conjugate{W}}_{\times l'}
\end{align}
to $\complexnumbers$. A \emph{spinor} is an element of such a two-dimensional complex vector space $W$ equipped with an antisymmetric non-zero type $(0, 2, 0, 0)$ tensor $\tensor{\epsilon}{_A_B}$. Analogously to a metric, $\tensor{\epsilon}{_A_B}$ maps spinors into dual spinors by $\tensor{\xi}{^A} \mapsto \tensor{\epsilon}{_A_B} \tensor{\xi}{^A}$.


Why are we interested in spinors?


\subsection{Spinors in curved spacetime}%
\label{sub:spinors_in_curved_spacetime}

It turns out that the appropriate space to define spinors on is a fiber bundle\todo{elaborate?}.If $G$ is a Lie group and $M$ is a manifold, then a fiber bundle is a manifold which locally looks like $G \times M$. The following definition captures this concept, too, in a universal property.\todo{maybe $\pi$ isn't a up now in the same way as for tensors}
\begin{definition*}
	Let $G$ be a Lie group and let $M$ and $B$ be manifolds. Let $\phi \from G \times B \to B$ be a \emph{free left action}, i.e.\ $g \mapsto \phi(g, \cdot)$ is a homomorphism from $G$ to the set of diffeomorphisms on $B$. Then $(B, G, M, \phi)$ is a \emph{principal fibre bundle} if
	\begin{enumerate}
		\item There is a bijection between orbits of $\phi(g, \cdot)$ and the points of $M$ such that the projection $\pi \from B \to M$ mapping each element in $B$ to the element in $M$ corresponding to its orbit under $\phi(g, \cdot)$ is smooth.
		\item For each $x \in M$, there is some neighbourhood $U$ of $x$ such that there exists a diffeomorphism $\psi \from \psi^{-1}(U) \to G \times U$ satisfying $\psi(\phi(g', \cdot)) \from (g, x) \mapsto (g' g, x)$.
	\end{enumerate}
\end{definition*}
\todo{maybe use the wikipedia definition}


Maybe talk about the stuff on p.\ 369 and forward in Wald about derivatives of spinors?


