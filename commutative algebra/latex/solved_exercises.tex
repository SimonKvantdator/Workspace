%! TEX root = /home/simon/Documents/masterarbete/dagbok/main.tex
\documentclass[swedish, a4paper, 11pt]{report}


%%%%%%%%%%% LANGUAGE %%%%%%%%%%%

% For correct hyphenation in swedish
\usepackage[T1]{fontenc}

% For interpreting non-ASCII characters
\usepackage[utf8]{inputenc}

% International language support
% Fetches language from documentclass options. Most other packages do this as well
\usepackage{babel}


%%%%%%%%%%% FORMAL STUFF %%%%%%%%%%%

% Smaller margins
\usepackage[margin=2.5cm]{geometry}

% Fancy chapter headers
\usepackage{titlesec}
\titleformat{\chapter}{\normalfont\huge}{\thechapter.}{20pt}{\huge\it}

% Dates & time
\usepackage[yyyymmdd]{datetime} % Useful when referencing websites
\renewcommand{\dateseparator}{-} % ISO 8601 date format

% What to display in table of contents
\setcounter{tocdepth}{1}
\setcounter{secnumdepth}{2}

% Lists
\usepackage{enumerate} % Determines the style in which the counter is printed
\usepackage{enumitem} % Provides user control over the layout of the three basic list environments

% Citing & bibliography
\usepackage{csquotes} % For \enquote command for proper quotation marks, also biblatex recommends this
\usepackage[numbers]{natbib}


%%%%%%%%%%% GRAPHICS %%%%%%%%%%%

\usepackage{graphics,color,xcolor}

% Figures
\usepackage{epsfig} % Solves some problems in \includegraphics{<.eps-file>}
\usepackage{graphicx} % More options for \includegraphics
\usepackage{wrapfig} % Figure environment that lets text wrap around figure
\usepackage{float} % Figure placement
\usepackage{caption} % More options for \caption
\usepackage{subcaption} % Subfigures

% Tikz
\usepackage{tikz}
\usepackage{pgf,pgfplots} % Pgfplot
\pgfplotsset{compat=1.15}

% För alduslöv
\usepackage{pifont}


%%%%%%%%%%% PHYSICS %%%%%%%%%%%

% SI units
\usepackage{siunitx}
\DeclareSIUnit\clight{\text{$c$}} % redefine from c_0 to c
\DeclareSIUnit\byte{B}

% Physics macros
\usepackage{physics} % Defines lots of nice commands like \derivative, \norm, \evaluated, etc. It is recommended to use these as much as possible for nice spacing and readable LaTeX code.
\usepackage{braket} % Defines \bra, \ket, \braket, and \set
\usepackage{slashed} % For Feynman slash notation
\usepackage{simpler-wick} % Wick contractions (may require sty-file)
% \usepackage[compat=1.1.0]{tikz-feynman} % Feynman diagrams (has to be compiled with LuaTeX)
\usepackage{tensor} % Covariant index notation


%%%%%%%%%%% CODING %%%%%%%%%%%

% For nice code insertions
\usepackage{listings}
\definecolor{codegreen}{rgb}{0,0.6,0}
\definecolor{codegray}{rgb}{0.5,0.5,0.5}
\definecolor{codepurple}{rgb}{0.58,0,0.82}
\definecolor{backcolour}{rgb}{0.95,0.95,0.92}
\lstdefinestyle{mystyle}{
    backgroundcolor=\color{backcolour},   
    commentstyle=\color{codegreen},
    keywordstyle=\color{magenta},
    numberstyle=\tiny\color{codegray},
    stringstyle=\color{codepurple},
    basicstyle=\ttfamily\footnotesize,
    breakatwhitespace=false,         
    breaklines=true,                 
    captionpos=b,                    
    keepspaces=true,                 
    numbers=left,                    
    numbersep=5pt,                  
    showspaces=false,                
    showstringspaces=false,
    showtabs=false,
    tabsize=4
}
\lstset{style=mystyle}


%%%%%%%%%%% MATHEMATICS %%%%%%%%%%%

% AMS packages
\usepackage{amsmath,amsfonts,amsthm,amssymb}

% Theorem and proof environments
\iflanguage{swedish}{
    \newtheorem{theorem}{Sats}
    \newtheorem*{theorem*}{Sats}
    \newtheorem{proposition}{Proposition}
    \newtheorem*{proposition*}{Proposition}
    \newtheorem{corollary}{Följdsats}[theorem]
    \newtheorem{corollary*}{Följdsats}
    \newtheorem{lemma}{Lemma}
    \newtheorem*{lemma*}{Lemma}
    \theoremstyle{definition}
    \newtheorem{definition}{Definition}
    \newtheorem*{definition*}{Definition}
}{}
\iflanguage{english}{
    \newtheorem{theorem}{Theorem}
    \newtheorem*{theorem*}{Theorem}
    \newtheorem{proposition}{Proposition}
    \newtheorem*{proposition*}{Proposition}
    \newtheorem{corollary}{Corollary}[theorem]
    \newtheorem{corollary*}{Corollary}
    \newtheorem{lemma}{Lemma}
    \newtheorem*{lemma*}{Lemma}
    \theoremstyle{definition}
    \newtheorem{definition}{Definition}
    \newtheorem*{definition*}{Definition}
}{}

% Better version of the \not command
\usepackage{cancel}

 % Does polynomial division for you
\usepackage{polynom}

% Vectors are upright boldface. I think this definition is better than the physics package's \vectorbold.
\let\Vec\undefined % We use \vec w/ lowercase v
\renewcommand*{\vec}[1]{{\boldsymbol{\mathrm{#1}}}}

% Bar, tilde, and hat that scales with what is under them. Basically I just want these to have consistent names
\let\mathbar\overline
\let\mathtilde\widetilde
\let\mathhat\widehat

% Redefine \exp
% Errors occur if this definition is made before some of the packages are loaded
\let\oldexp\exp
\newcommand*{\Exp}[1]{\oldexp{#1}}
\renewcommand{\exp}[1]{\mathrm{e}^{#1}}

% Main number systems
\newcommand{\naturals}{\mathbb{N}}
\newcommand{\integers}{\mathbb{Z}}
\newcommand{\rationals}{\mathbb{Q}}
\newcommand{\reals}{\mathbb{R}}
\newcommand{\compexnumbers}{\mathbb{C}}

% Some of my own shorthands for correct spacing in math environments
\def\divides{\mid} % Proper spacing of vertical bar in division x|y
\def\from{\colon} % Proper spacing of colon in functions f:A→ B
\newlength\mylen % Isomorphic \mapsto
\settowidth\mylen{$\longleftrightarrow$}
\newcommand{\mapsbetween}{\longleftrightarrow\kern - 0.5\mylen\vline height 1.2ex depth -0.0pt\kern0.5\mylen}
\newcommand{\suchthat}{\qq{s.th.}}
\def\definedas{\coloneqq}
\def\defines{\eqqcolon}

\newcommand*{\transpose}[1]{#1^{\!\mathsf{T}}}
\renewcommand*{\complement}[1]{#1^{\mathsf{C}}}
% \newcommand{\conjugate}[1]{\mathbar{#1}}
\newcommand*{\conjugate}[1]{#1^*}
\newcommand*{\hermitianconjugate}[1]{#1^\dag}
\newcommand*{\inverse}[1]{#1^{-1}}

\newcommand*{\closure}[1]{\mathbar{#1}} % Closure of a set
\def\union{\cup}
\newcommand{\Union}{\bigcup\limits}
\def\intersection{\cap}
\newcommand{\Intersection}{\bigcap\limits}

% Lie-groups & algebras, i.g. SU(n)
\newcommand*{\algebra}[2]{\mathfrak{\MakeLowercase{#1}}\left(#2\right)}
\newcommand*{\group}[2]{\mathrm{\MakeUppercase{#1}}\left(#2\right)}


%%%%%%%%%%% MISCELLANEOUS %%%%%%%%%%%

% In-pdf comments through \todo command
\setlength{\marginparwidth}{2cm} % Silence warning about margin size
\usepackage{todonotes}

% Clickable links and refs
\usepackage[hidelinks]{hyperref} 

% Cleverref automatically detects if you are referencing a figure, table, or equation etc
% Cleverref has to be loaded last I think, after babel and hyperref etc
\usepackage[noabbrev, nameinlink]{cleveref}
\crefname{equation}{}{}
\iflanguage{swedish}{ % Tell cleverref to use Oxford comma
	\newcommand{\creflastconjunction}{, och\nobreakspace}
}{}
\iflanguage{english}{
	\newcommand{\creflastconjunction}{, and\nobreakspace}
}{}


% Tag only referenced equations (this is not an ideal package, as it requires etextools, which is buggy and abandoned by its author)
\expandafter\def\csname ver@etex.sty\endcsname{3000/12/31}
\let\globcount\newcount
\usepackage{autonum}

% Intervals on the real line
\let\interval\undefined % To avoid name conflict with etextools
\usepackage{interval}
\intervalconfig{soft open fences}

% For writing \overset{text}&{=} in align environment
\usepackage{aligned-overset} 

\title{{\Huge \ding{166}}
	\parbox{0.85\textwidth}{
		\centering
		Solved exercises in Miles Reid's Undergraduate Commutative Algebra
	}
	{\Huge \ding{166}}}
\author{Simon Stefanus Jacobsson}
\date{\today}

	
\begin{document}
\maketitle

\subsection{}

Let $A$ be a ring and consider the polynomial ring $A[T]$. Prove that $T$ is not a zero-divisor in $A[T]$. Generalize the argument to prove that a monic polynomial
\begin{align}
	f = T^n + a_{n - 1} T^{n - 1} + \ldots + a_0
\end{align}
is not a zero-divisor in $A[T]$.

\subsubsection*{Solution}
Let $g = a_n T^n + \ldots + a_1 T + a_0 \in A[T]$ and let $g T = 0$. Then
\begin{align}
	a_n T^{n + 1} + \ldots + a_1 T^2 + a_0 T = 0
\end{align}
implies $a_n, \ldots, a_1, a_0 = 0$ by uniqueness of polynomial coefficients.

Similarly, if $g (T^n + b_{n-1} T^{n - 1} + \ldots + b_1 T + b_0)$, the $T^{2 n}$-term gives that $a_n = 0$, the $T^{2 n -1}$-term gives that $a_{n - 1} = 0$, and so on. Hence $g = 0$ and any monic polynomial will not be a zero-divisor.


\subsection{}

Let $A$ be a ring, $a \in A$, and $f \in A[T]$. Prove that there exists an expression $f = (T - a) q + r$ with $q \in A[T]$ and $r \in A$. [Hint: subtract off a suitable multiple of $(T - a)$ to cancel the leading term, then use induction on $\operatorname{deg} f$.] By substituting $T = a$, show that $r = f(a)$. (this result is often called the \emph{remainder theorem} in algebra textbooks.)

\subsubsection*{Solution}

For induction, assume it holds for $\operatorname{deg} f = p$. Then, for $f = c_{p + 1} T ^{p + 1} + \ldots + c_0$, we have $\operatorname{deg} \left( f - c_{p + 1} (T - a) \right) = p$. Hence
\begin{align}
	f - c_{p + 1} (T - a) &= (T - a) q' + r \quad \text{ for some } q' \in A[T] \text{ and } r \in A\\
	f &= (T - a) (q' + c_{p + 1}) + r
\end{align}
and hence it is true for $\operatorname{deg} f = p + 1$ (with $q = q' + c_{p + 1}$).

For $\operatorname{deg} f = 0$, it is obviously true with $q = 0$ and $r = f$.


\addtocounter{subsection}{1}


\subsection{}

Let $A[T]$ be the polynomial ring over a ring $A$, and let $B$ be a ring. Suppose that $\varphi \from A \to B$ is a given ring homomorphism; show that ring homomorphisms $\psi \from A[T] \to B$ extending $\varphi$ are in one-to-one correspondence with elements in $B$.

\subsubsection*{Solution}

This is clear since in
\begin{align}
	\psi\left( f(T) \right) &= \psi\left( c_n T^n + \ldots + c_1 T + c_0 \right)\\
	&=  c_n \psi(T)^n + \ldots + c_1 \psi(T) + c_0,
\end{align}
if we know $\psi(T)$, we know the whole expression, and $\psi(T)$ can be any element in $B$.

\addtocounter{subsection}{2}

\subsection{}
TODO





\section{}
\subsection{}

Give an example of a ring $A$ and ideals $I$, $J$ such that $I \cup J$ is not an ideal; in your example, what is the smallest ideal containing $I$ and $J$?

\subsubsection*{Solution}

Consider $(6)$ and $(10)$ in $\Z$. $10 \in (10)$ and $6 \in (6)$, but $10 + 6 = 16 \not\in (10) \union (6)$, so $(10) \union (6)$ is not even a subring of $\Z$. The smallest ideal containing $(10) \union (6)$ is $\left( \operatorname{gcd}(6, 10) \right) = (2)$.


\subsection{}

The \emph{product} of two ideals $I$ and $J$ is the set of all sums $\sum_i f_i g_i$ with $f_i \in I$ and $g_i \in J$. Give an example in which $I J \neq I \intersection I$.

\subsubsection*{Solution}

$I = J = (2)$ gives $I J = (4) \neq (2) = I \intersection J$.


\subsection{}

Let $A = k[X, Y] / (X Y)$. Show that any element of $A$ has a unique representation in the form
\begin{align}
	a + f(X) X + g(Y) Y \qq{with} a \in k \text{, } f \in k[X] \text{, and } g \in k[Y].
\end{align}
How do you multiply two such elements?

Prove that $A$ has exactly two minimal prime ideals. If possible, find ideals $I$, $J$, and $K$ to contradict each of the following statements:
\begin{enumerate}
	\item $I J = I \intersection J$
	\item $(I + J) (I \intersection J) = I J$
	\item $I \intersection (J + K) = (I \intersection J) + (I \intersection K)$.
\end{enumerate}

\subsubsection*{Solution}

TODO


\subsection{}

Two ideals $I$ and $J$ are \emph{strongly coprime} if $I + J = A$. Check that this is the usual notion for coprime $A = \Z$ or $k[X]$. Prove that if $I$ and $J$ are strongly coprime, then
\begin{align}
	I J = I \intersection J \qq{and} A / I J \sim (A / I) \times (A / J).
\end{align}
Prove also that if $I$ and $J$ are strongly coprime then so are $I^n$ and $J^n$ for $n \geq 1$.

\subsubsection*{Solution}

Since $\Z$ is a PID, let $I = (a)$ and $J = (b)$. Then
\begin{align}
	I + J = (a) + (b) = (a, b) = \left( \operatorname{gcd}(a, b) \right) =
	\begin{cases}
		(1) = A &\text{ if $a$ and $b$ coprime in the usual sense}\\
		\text{not } (1) &\text{ if $a$ and $b$ not coprime in the usual sense}
	\end{cases}.
\end{align}
Similarly for $k[X]$.

In general, $I J \subset I \intersection J$ since $I J \subset I A = I$ and $I J \subset A J = J$. TODO


\subsection{}

Let $\varphi \from A \to B$ be a ring homomorphism. Prove that $\varphi^{-1}$ takes prime ideals of $B$ to prime ideals of $A$. In particular, if $A \subset B$, and $P$ is a prime ideal of $B$ then $A \intersection P$ is a prime ideal of $A$.

\subsubsection*{Solution}

If $P$ is a prime ideal in $B$, then $\varphi^{-1}(P)$ is an ideal since if $\varphi(a) \in P$, then $\varphi(a b) = \varphi(a) \varphi(b) \in P$ by $P$ being an ideal. Furthermore, $\varphi^{-1}(P)$ is prime since if $\varphi(a)$, $\varphi(b) \in P^\complement$, then $\varphi(a b) = \varphi(a) \varphi(b) \in P^\complement$ by $P$ being prime.



\subsection{}

Prove or give a counterexample to
\begin{enumerate}
	\item the intersection of two prime ideals is prime \label{item:intersection_of_primes_is_prime}
	\item the ideal $P_1 + P_2$ generated by two prime ideals $P_1$ and $P_2$ is prime \label{item:sum_of_primes_is_prime}
	\item if $\varphi \from A \to B$ is a ring homomorphism then $\varphi^{-1}$ takes maximal ideals of $B$ to maximal ideals of $A$ \label{item:phi_inv_takes_maximals_to_maximals}
	\item the map $\varphi^{-1}$ of Proposition 1.2 (quotient homomorphism) takes maximal ideals of $A / I$ to maximal ideals of $A$. \label{item:prop1.2_phi_inv_takes_maximals_to_maximals}
\end{enumerate}

\subsubsection*{Solution}

\Cref{item:intersection_of_primes_is_prime} is false since $(2)$ and $(3)$ in $\Z_6$ is a counterexample. $(2) \intersection (3) = (0)$ is not prime in $\Z_6$ since $[2] [3] = [0]$.

\Cref{item:sum_of_primes_is_prime} is false since $(2)$ and $(x^2 + 3)$ in $\Z[X]$ is a counterexample. Both $2$ and $x^2 + 3$ are irreducible in $\Z[X]$, but $(2) + (x^2 + 3) = (2, x^2 + 3)$ is not prime since $x^2 - 1 \in (2, x^2 + 3)$ but $x^2 - 1 = (x + 1) (x - 1)$ while neither of those factors are in $(2, x^2 + 3)$.

\Cref{item:phi_inv_takes_maximals_to_maximals,item:prop1.2_phi_inv_takes_maximals_to_maximals} are also false by the same counterexample. Let $A = \Z[X, Y]$, let $I = (X - 2, Y - 3)$, and let $\varphi$ be the quotient homomorphism. We can identify $A / I$ with $\Z$ and say that $\varphi \from X \mapsto 2$ and $Y \mapsto 3$. Then $(2)$ is maximal in $\Z$ but $\varphi^{-1}\left( (2) \right) = (2, X)$ is not maximal in $\Z[X, Y]$.
















\end{document}